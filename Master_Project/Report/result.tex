\chapter{Results and Discussion} \label{chap_results}

This chapter will present the results from performance testing our purposed architecture. We will examine performance for both speedup and power efficiency. In the second section we will discuss and analyze the provided results. rewrite this stuff...

\section{Results}

\subsection{Hardware Resources}

The architecture described in Chapter \ref{chap_method} was prototyped on an \textit{Avnet Zedboard}, containing a \textit{Xilinx Zyngq-7020 All Programmable System-On-Chip} (SoC). The SoC contains two ARM Cortex-A9 processors, and a Artix-7 FPGA. 

The main reason for choosing this system was that it contains four DMA channels, and 220 DSP slices, which should have allowed us to run four fully saturated accelerators in parallel. Unfortunately resource constraint for our architecture turned out to be \textit{look-up tables} (LUTs)

In this prototype we were only able to use one of the two ARM processors for controlling the accelerator(s) and processing the layers that were not accelerated. Preferably we should have used both for processing layer C5 and F6, but we were unable to do so due to time constraints. 


\subsection{Performance}

In order to determine the execution speed and power efficiency of our system we have compared it to the ARM Cortex-A9 CPU on the Zedboard and an ASUS X550JK laptop with a Intel Core i7 4710HQ CPU. Both CPUs ran the pure software implementation of the CNN, while our system used a combination of hardware and software, as described in \ref{chap_method}. We ran our own system with three different configurations:

\begin{itemize}
	\item Accelerating layer C1 and S2. 
	\item Accelerating layer C1, S2, C3 and S4.
	\item Accelerating layer C1, S2, C3 and S4. In addition, the input images was preprocessed from 32-bit floating point to Q16:16 fixed point. 
\end{itemize}

In order to determine the energy efficiency of the different systems we used the metric \textit{images/Watt}, i.e. number of images processed per Watt. Note that these images are $ 32 \times 32 $, and thus processing one image corresponds to 331104 multiply-and-accumulate operations. We also included a metric for measuring execution speed, using images/second. Despite power efficiency being the main focus of this assignment, execution speed can be interesting for several applications and is closely related to power usage. 

The measurements were done by timing the processing of 10 000 images from the MNIST dataset, while measuring the power consumation. 

Total board power was determined by measuring over pin 1 and 2 on J21 on the Zedboard during execution. With the FPGA programmed and the accelerator activated the board measured to 4.68 W, while the ARM processor alone measured to 4.32 W. We were unable to measure the power consumption of the laptop directly, and therefore used the power estimation provided by ASUS, being 120 W \cite{ASUS}.  

The results can be seen in Figure \ref{fig_results_all_layers}.

\begin{figure}[h!]
	\centering
	\includegraphics[width=1.0\textwidth]{Figures/Results/results_all_layers}
	\caption{The execution speed and power efficiency measured in number of images processed.}
	\label{fig_results_all_layers}
\end{figure}

We also decided to perform the same measurements when processing only the layers we have hardware accelerated, i.e. C1, S2, C3 and S4. This allowed us to compare the accelerator more directly against the pure software implementations, since layer C5 and F6 shows to be the major bottleneck when processed on the ARM processor. 

